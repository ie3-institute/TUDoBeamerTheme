% This example is meant to be compiled with lualatex or xelatex
% The theme itself also supports pdflatex
\PassOptionsToPackage{unicode}{hyperref}
\documentclass[aspectratio=1610, 9pt]{beamer}

% Load packages you need here
\usepackage{polyglossia}
\setmainlanguage{german}

\usepackage{csquotes}
    

\usepackage{amsmath}
\usepackage{amssymb}
\usepackage{mathtools}

\usepackage{hyperref}
\usepackage{bookmark}

% load the theme after all packages

\usetheme[
  showtotalframes, % show total number of frames in the footline
]{tudo}

% Put settings here, like
\unimathsetup{
  math-style=ISO,
  bold-style=ISO,
  nabla=upright,
  partial=upright,
  mathrm=sym,
}

\title{\LaTeX-Beamer-Theme der TU~Dortmund}
\subtitle{A nice template for presentations of ie\textsuperscript{3} institute}
\author[Prof.~K.-D.~Brokkoli]{Univ.-Prof.~Dr.~{}rer.~hort.~Klaus-Dieter~Brokkoli}
% \author[Univ.-Prof.~Dr.~rer.~hort.~K.-D.~Brokkoli]{Univ.-Prof.~Dr.~{}rer.~hort.~Klaus-Dieter~Brokkoli}
\institute[ie\textsuperscript{3}]{Institute of Energy Systems, Energy Efficiency and Energy Economics}

\begin{document}

\maketitle

\begin{frame}{How to use}
  Props to contributors \href{https://github.com/maxnoe}{Maximilian N\"othe} and \href{https://github.com/bernharddick}{Bernhard Dick} for the ground-laying work on this theme!
  
  \medskip
  To install this theme on your system, at least \texttt{beamerthemetudo.sty} and the provided directory \texttt{logos} have to be in a place, where \LaTeX is able to find those packages.
  This is one of either:
  \begin{itemize}
    \item \texttt{TEXMFHOME/tex/latex/tudobeamertheme}. You may find the location of \texttt{TEXMFHOME} with \texttt{kpsewhich --var-value TEXMFHOME}, typically it's at \texttt{\$HOME/texmf}.
    \item The same directory, your document to compile is in.
    \item Any other place, that is added to \texttt{TEXINPUTS}.
  \end{itemize}
  
  \medskip
  You may install the template with this single line command:\\
  \texttt{\footnotesize\$ cd `kpsewhich --var-value TEXMFHOME` \&\& git clone https://github.com/ie3-institute/tudobeamertheme}

  \medskip
  Additional information about \LaTeX and Beamer may be found at:
  \begin{itemize}
    \item Extensive \LaTeX course by PeP et Al. \\
      \url{http://toolbox.pep-dortmund.org/notes}
    \item \LaTeX-Beamer documentation:\\
    \url{http://www.ctan.org/pkg/beamer}
  \end{itemize}
\end{frame}

\begin{frame}{Additional Features}
  There are two options available to alter the appearance of your slides:

  \medskip
  If you place \texttt{logos/chair.pdf} in one of the above mentioned places, the abbreviation of your chair is replaced by this logo.

  \medskip
  If you place \texttt{images/tudo-title-1.jpg} to \texttt{images/tudo-title-4.jpg} in one of the above mentioned places, the first, thrid, fifth and seventh box on the title page will be replaced by this picture.
\end{frame}

\begin{frame}{Fonts}
  As of Coporate Design of TU Dortmund university, \enquote{Akkurat Office} is the foreseen font to use.

  If that one isn't available, alternatively \enquote{Fira Sans} is used.
  So make sure, you have at least one of both installed.
  
  For mathematical typesetting, \enquote{Fira Math} is the font of choice, if you compile using \texttt{xelatex} or \texttt{lualatex}.
\end{frame}
\end{document}
